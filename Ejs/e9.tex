\item Se carga un dado de manera que los números pares tienen el doble de probabilidad de salir que los impares; los pares son igualmente probables entre sí, y lo mismo sucede con los impares. Se arroja el dado una vez. Hallar la probabilidad que:
    \begin{enumerate}
        \item Aparezca un número par.\e\\
            Tenemos que los impares son la mitad de probables que los pares. Si consideramos los eventos
            \begin{center}
                $A=$ sale un número par\\
                $B=$ sale un número impar
            \end{center}
            Entonces\[P(A)=2P(B)\]
            Además, el espacio muestral es\[\Omega=\{1,2,3,4,5,6\}\]
            y se sabe que\[A=\{2,4,6\}\qquad B=\{1,3,5\}\]
            Por el axioma 3\[P(A)=p(2)+p(4)+p(6)\qquad P(B)=p(1)+p(3)+p(5)\]
            Y como cada par e impar tienen la misma probabilidad entre sí\[P(A)=6c\qquad P(B)=3c\]
            Al ser $A$ y $B$ disjuntos\[P(\Omega)=P(A\cup B)=P(A)+P(B)=9c=1\to c=\frac{1}{9}\]
            Por lo tanto\[P(A)=6\cdot\frac{1}{9}=\frac{6}{9}\]
        \item Aparezca un número impar.
            \[P(B)=3\cdot\frac{1}{9}=\frac{3}{9}\]
        \item Aparezca un número primo impar.\e\\
            Los números primos impares que pueden salir en un dado son 3 y 5. Como los impares son igual de probables y $P(B)=\frac{3}{9}\to p(1)=p(3)=p(5)=\frac{1}{9}$\[\to p(3)+p(5)=\frac{2}{9}\]
    \end{enumerate}