\item Sean $E_1,\dots,E_n$ $n\geq2$, eventos cualesquiera de $S$.
    \begin{enumerate}
        \item Probar $P\left(\bigcup\limits_{i=1}^n E_i\right)\leq \sum\limits_{i=1}^nP(E_i)$\e\\
            Se define
            \begin{align*}
                B_1&=E_1\\
                B_2&=E_2-E_1\\
                B_3&=E_3-(E_1\cup E_2)\\
                \vdots\\
                B_n&=E_n-\left(\bigcup_{i=1}^{n-1}E_i\right)
            \end{align*}
            Sean $B_k,B_m, k>m$. Entonces $B_k=E_k-\left(\bigcup\limits_{i=1}^{k-1}E_i\right), B_m=E_m-\left(\bigcup\limits_{i=1}^{m-1}E_i\right)$
            \begin{align*}
                B_k\cap B_m&=E_k-\left(\bigcup_{i=1}^{k-1}E_i\right)\cap E_m-\left(\bigcup_{i=1}^{m-1}E_i\right)&&\text{Def}\\
                &=E_k\cap\left(\bigcup_{i=1}^{k-1}E_i\right)^c\cap E_m\cap\left(\bigcup_{i=1}^{m-1}E_i\right)^c&&\text{Resta}\\
                &=E_k\cap\left(\bigcap_{i=1}^{k-1}E_i^c\right)\cap E_m\cap\left(\bigcap_{i=1}^{m-1}E_i^c\right)&&\text{De Morgan}\\
                &=E_k\cap\left(\bigcap_{i=1}^{m-1}E_i^c\right)\cap\left(\bigcap_{i=m}^{k-1}E_i^c\right)\cap E_m\cap\left(\bigcap_{i=1}^{m-1}E_i^c\right)&&k>m\\
                &=E_k\cap\left(\bigcap_{i=1}^{m-1}E_i^c\right)\cap E_m^c\cap\left(\bigcap_{i=m+1}^{k-1}E_i^c\right)\cap E_m\cap\left(\bigcap_{i=1}^{m-1}E_i^c\right)\\
                &=\varnothing&&E_m^c\cap E_m\cap A=\varnothing\quad \forall A
            \end{align*}
            Si $m>k$, se pueden invertir los lugares y se llega al mismo resultado. Como consecuencia:
            \begin{equation}
                B_k\cap B_m=\varnothing\qquad\forall k\neq m
            \end{equation}
            Es decir, $B_k$ y $B_m$ son eventos independientes. Por otra parte, dado un $n$ arbitrario
            \[
                B_n=E_n-\left(\bigcup_{i=1}^{n-1}E_i\right)=E_n\cap\left(\bigcup_{i=1}^{n-1}E_i\right)^c\subseteq E_n
            \]
            \begin{equation}
                P(B_n)\leq P(E_n)
            \end{equation}
            Además, vemos que si $n=2$
            \begin{align*}
                \left(\bigcup_{i=1}^2 B_i\right)&=E_1\cup E_2-E_1\\
                &=E_1\cup E_2\cap E_1^c&&\text{Resta}\\
                &=(E_1\cup E_2)\cap(E_1\cup E_1^c)&&\text{Distributiva}\\
                &=(E_1\cup E_2)\cap\Omega\\
                &=E_1\cup E_2&&\text{Intersección con el universo}\\
                &=\left(\bigcup_{i=1}^2 E_i\right)
            \end{align*}
            Suponemos que funciona para un $n$ arbitrario y comprobamos para $n+1$
            \begin{align*}
                \bigcup_{i=1}^{n+1} B_i&=\left(\bigcup_{i=1}^{n} B_i\right)\cup B_{n+1}\\
                &=\left(\bigcup_{i=1}^{n} E_i\right)\cup B_{n+1}&&\text{Hipótesis Inductiva}\\
                &=\left(\bigcup_{i=1}^{n} E_i\right)\cup E_{n+1}-\left(\bigcup_{i=1}^{n} E_i\right)&&\text{Def }B_{n+1}\\
                &=\left(\bigcup_{i=1}^{n} E_i\right)\cup \left[E_{n+1}\cap\left(\bigcup_{i=1}^{n} E_i\right)^c\right]&&\text{Resta}\\
                &=\left[\left(\bigcup_{i=1}^{n} E_i\right)\cup E_{n+1}\right]\cap\left[\left(\bigcup_{i=1}^{n} E_i\right)\cup\left(\bigcup_{i=1}^{n} E_i\right)^c\right]&&\text{Distributiva}\\
                &=\left(\bigcup_{i=1}^{n+1} E_i\right)\cap \left[\left(\bigcup_{i=1}^{n} E_i\right)\cup\left(\bigcap_{i=1}^{n} E_i^c\right)\right]&&\text{De Morgan}\\
                &=\left(\bigcup_{i=1}^{n+1} E_i\right)\cap\left[\left(\left(\bigcup_{i=1}^{n} E_i\right)\cup E_1^c\right)\cap\dots\cap\left(\left(\bigcup_{i=1}^{n} E_i\right)\right)\cup E_n^c\right]&&\text{Distributiva}\\
                &=\left(\bigcup_{i=1}^{n+1} E_i\right)\cap\left[\Omega\cap\dots\cap\Omega\right]&&\text{En el término }j:E_j\cup E_j^c=\Omega\\
                &=\bigcup_{i=1}^{n+1} E_i&&\text{Intersección universo}
            \end{align*}
            Por lo tanto
            \begin{equation}
                \bigcup_{i=1}^{n} E_i=\bigcup_{i=1}^{n} B_i\qquad\forall n
            \end{equation}
            Con esto podemos ver que
            \begin{align*}
                P\left(\bigcup\limits_{i=1}^n E_i\right)&=P\left(\bigcup\limits_{i=1}^n B_i\right)&&\text{Por 3}\\
                &=\sum_{i=1}^nP(B_i)&&\text{Por 1}\\
                &\leq \sum\limits_{i=1}^nP(E_i)&&\text{Por 2}
            \end{align*}
        \item Probar que si cada $E_i$ es un evento casi seguro (es decir que $P(E_i)=1$), entonces $E_1\cap\dots\cap E_n$ es un evento casi seguro.\e\\
            Está en la teoría.
    \end{enumerate}