\item Sean $A$ y $B$ dos eventos tales que $P(A)=0.5,P(A\cap B)=0.2$ y $P(A\cup B)=0.7$. Hallar
    \begin{enumerate}
        \item $P(B)$\e\\
            Del principio de inclusión-exclusión se sabe que $P(A\cup B)=P(A)+P(B)-P(A\cap B)$. Entonces,\begin{center}
                $P(B)=P(A\cup B)+P(A\cap B)-P(A)$\\
                $\to P(B)=0.7+0.2-0.5$\\
                $\to P(B)=0.4$    
            \end{center}
        \item $P$(ocurra exactamente uno de los dos eventos)\e\\
            La probabilidad de que suceda alguno de los dos eventos está dada por $P(A\cup B)$. A ésta hay que restarle la probabilidad de que sucedan los dos simultáneamente, Entonces\[P(\text{ocurra exactamente uno de los dos eventos})=P(A\cup B)-P(A\cap B)=0.5\]
        \item $P$(no ocurra ninguno de los eventos)\e\\
            Se sabe que la probabilidad de que ocurra al menos uno es $P(A\cup B)$. Entonces\[P(\text{no ocurra ninguna de los eventos})=1-P(A\cup B)=0.3\]
    \end{enumerate}