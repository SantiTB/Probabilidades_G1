\item Asociar un espacio muestral a cada uno de los siguientes experimentos aleatorios.
    \begin{enumerate}
        \item Lanzar tres veces al aire una moneda y observar el lado que cae hacia arriba.\e\\
            El espacio muestral es el conjunto de resultados posibles de un experimento. En este caso particular, al tirar una moneda la misma puede caer con cara o cruz hacia arriba. Podemos denotar los siguientes eventos:
            \begin{center}
                A: Sale cara\\
                B: Sale cruz
            \end{center}
            Entonces tenemos que:\[\Omega=\{AAA,AAB,ABA,ABB,BAA,BBA,BAB,BBB\}\]
        \item Lanzar tres veces al aire una moneda y observar el número total de caras.\e\\
            Puede suceder que en los tres tiros no salga cara, también puede caer una vez, dos veces, e incluso tres. Por lo tanto:\[\Omega=\{0,1,2,3\}\]
        \item Una urna contiene 2 bolillas blancas y una negra. Se sacan 2 bolillas al azar simultáneamente y se anotan los colores.\e\\
            Como estoy anotando los colores, no se distingue entre las dos bolillas blancas. Considerando los eventos
            \begin{center}
                N: negra\\
                B: blanca
            \end{center}
            se tiene que\[\Omega=\{BB,BN\}\]
        % Item D ------------------------------------------------------------------------------------------------------
        \item Idem que en el inciso anterior pero con reemplazo.\e\\
            En este caso, al poder reemplazar, está la posibilidad de sacar dos veces la negra.\[\Omega=\{BB,BN,NB,NN\}\]
        % Item E ------------------------------------------------------------------------------------------------------
        \item Se colocan al azar tres bolillas diferentes en tres urnas diferentes, pudiéndose poner más de una bolilla por urna.\e\\
            Si representamos cada resultado como \[(U_{B_1},U_{B_2},U_{B_3})\]
            siendo $U_{B_i}$ la urna en donde se encuentra la bolilla $i$. Entonces
            \begin{center}
                $\Omega=\{(1,1,1),(1,1,2),(1,1,3),(1,2,1),(1,2,2),(1,2,3),(1,3,1),(1,3,2),(1,3,3),$\\
                $(2,1,1),(2,1,2),(2,1,3),(2,2,1),(2,2,2),(2,2,3),(2,3,1),(2,3,2),(2,3,3),$\\
                $(3,1,1),(3,1,2),(3,1,3),(3,2,1),(3,2,2),(3,2,3),(3,3,1),(3,3,2),(3,3,3)\}$
            \end{center}
            Como hay un número considerable de resultados, quizás es más conveniente expresar el espacio muestral por comprensión:
            \[\Omega=\{(i,j,k):i,j,k=1,2,3\}\]
        % Item F ------------------------------------------------------------------------------------------------------
        \item Se arroja una moneda; si sale cara se arroja un dado, si sale ceca se lanzan dos dados.\e\\
            Si denominamos los siguientes eventos:
            \begin{center}
                A: cara\\
                B: ceca
            \end{center}
            \[\Omega=\{(A,i):i\in\mathbb{N}/i<7\}\cup\{(B,i,j):i,j=1,2,3,4,5,6\}\]
        % Item G ------------------------------------------------------------------------------------------------------
        \item Un viajante debe visitar cinco ciudades y traza su itinerario.\e\\
            Supongamos que las ciudades las enumeramos del 1 al 5.\[\Omega=\{i_1,i_2,i_3,i_4,i_5:i_k\in[1;5],i_m\neq i_n\forall m\neq n\}\]
        % Item H ------------------------------------------------------------------------------------------------------
        \item Los artículos provenientes de una línea de producción se clasifican en defectuosos (D) y no defectuosos (N). Se observan artículos y se anota su condición. Este proceso se continúa hasta que se produzcan dos artículos defectuosos consecutivos o hasta que se hayan verificado cuatro artículos cualesquiera.
            \[\Omega=\{DD,NDD,NNDD,DNDD,DNDN,DNND,DNNN,NNNN,NDND,NDNN,NNDN,NNND\}\]
        % Item I ------------------------------------------------------------------------------------------------------
        \item Una caja con 12 lámparas tiene 4 unidades con filamentos rotos. Se las prueba hasta que se encuentre una
        quemada.\e\\
            Definimos los eventos:
            \begin{center}
                Q: quemada\\
                N: no quemada
            \end{center}
            Entonces,
            \[\Omega=\{N^iQ:0\leq i\leq8\}\]
        % Item J ------------------------------------------------------------------------------------------------------
        \item Un tanque de agua tiene una bomba cuyo motor se pone en funcionamiento automáticamente cuando el consumo
        hace que el volumen de agua baje hasta cierto nivel. Supongamos que esto puede ocurrir a lo sumo una vez al
        día. Cierto día se observa el motor durante 24 horas y se registra en qué instante ha comenzado a funcionar o
        si no lo ha hecho en todo el día.\e
            \begin{center}
                X: se prendió la bomba en el momento X\\
                N: no se prendió la bomba
            \end{center}
            Por lo tanto,
            \[\Omega=\{X,N\}\]
    \end{enumerate}