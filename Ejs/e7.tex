\item Un experimento aleatorio tiene tres resultados posibles: $a,b$ y $c$, con probabilidades $p,p^2$ y $p$ respectivamente. Hallar justificando apropiadamente el/los valores válidos de $p$.\e\\
    Tenemos que el espacio muestral es\[\Omega=\{a,b,c\}\]
    Entonces, por axioma 3:\begin{align*}
        P(\Omega)=1&=p(a)+p(b)+p(c)=p+p^2+p\\
        &\to p^2+2p-1=0\\
        &\to p=-1\pm\sqrt{2} 
    \end{align*}
    Sin embargo, $-1-\sqrt{2}$ implica una probabilidad negativa con lo cual\[p=-1+\sqrt{2}\]