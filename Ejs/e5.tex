\item Se realiza el siguiente experimento aleatorio: se lanza una moneda y un dado.
    \begin{enumerate}
        \item Definir un espacio muestral.\e\\
            La moneda puede resultar en cara o ceca, mientras que el dado en un natural menor o igual a 6.
            \begin{center}
                $X=$ cara\\
                $Y=$ ceca\\
            \end{center}
            Por lo tanto, el espacio muestral resulta ser\[\Omega=\{X1,X2,X3,X4,X5,X6,Y1,Y2,Y3,Y4,Y5,Y6\}\]
        \item Expresar explícitamente los siguientes sucesos:\\
            $A=$ "se obtiene un par y una cara".
            \[A=\{X2,X4,X6\}\]
            $B=$ "se obtiene un número primo".
            \[B=\{X2,X3,X5,Y2,Y3,Y5\}\]
            $C=$ "se obtiene un número impar y una ceca".
            \[C=\{Y1,Y3,Y5\}\]
        \item Encontrar expresiones para los siguientes eventos:
            \begin{enumerate}
                \item Sólamente ocurre $B$.\e\\
                    "Se obtiene un número primo"\e
                \item Ocurren tanto $A$ como $B$ pero no ocurre $C$.\e\\
                    "Se obtiene cara y un dos"\e
                \item Por lo menos dos ocurren.
                \item Ocurre uno y no más.
                \item No ocurren más de dos.
            \end{enumerate}
    \end{enumerate}